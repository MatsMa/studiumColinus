\documentclass[12pt]{article}
\usepackage[utf8]{inputenc}
\usepackage[T1]{fontenc}
\usepackage{amsmath}
\usepackage{amssymb}
\usepackage{amsthm}

\title{Lösungen zu Übungsaufgaben 02 \\ \small Gruppe: Mi 08-10 SR 2, Barbara Rieß}
\author{Linus Keiser}
\date{\today}

\newtheorem{theorem}{Satz}
\newtheorem{lemma}[theorem]{Lemma}

\begin{document}

\maketitle

\section*{Aufgabe 5}

\subsection*{(a) \( f^{-1}(V_1 \cup V_2) = f^{-1}(V_1) \cup f^{-1}(V_2) \)}

\begin{proof}
	Wir zeigen die Gleichheit der Mengen durch Nachweis der gegenseitigen Inklusion.

	\textbf{Zu zeigen:} \( f^{-1}(V_1 \cup V_2) \subseteq f^{-1}(V_1) \cup f^{-1}(V_2) \) und \( f^{-1}(V_1) \cup f^{-1}(V_2) \subseteq f^{-1}(V_1 \cup V_2) \).

	\textbf{1. Inklusion:} Sei \( x \in f^{-1}(V_1 \cup V_2) \). Dann gilt:
	\begin{align*}
		f(x) & \in V_1 \cup V_2                                              \\
		     & \Rightarrow f(x) \in V_1 \text{ oder } f(x) \in V_2           \\
		     & \Rightarrow x \in f^{-1}(V_1) \text{ oder } x \in f^{-1}(V_2) \\
		     & \Rightarrow x \in f^{-1}(V_1) \cup f^{-1}(V_2)
	\end{align*}

	\textbf{2. Inklusion:} Sei \( x \in f^{-1}(V_1) \cup f^{-1}(V_2) \). Dann gilt:
	\begin{align*}
		x & \in f^{-1}(V_1) \text{ oder } x \in f^{-1}(V_2)     \\
		  & \Rightarrow f(x) \in V_1 \text{ oder } f(x) \in V_2 \\
		  & \Rightarrow f(x) \in V_1 \cup V_2                   \\
		  & \Rightarrow x \in f^{-1}(V_1 \cup V_2)
	\end{align*}

	Da \( x \) in beiden Fällen zu \( f^{-1}(V_1 \cup V_2) \) gehört, folgt die Gleichheit der Mengen:
	\[ f^{-1}(V_1 \cup V_2) = f^{-1}(V_1) \cup f^{-1}(V_2) \]
\end{proof}

\subsection*{(b) \( f^{-1}(V_1 \cap V_2) = f^{-1}(V_1) \cap f^{-1}(V_2) \)}

\begin{proof}
	Wir zeigen die Gleichheit der Mengen durch Nachweis der gegenseitigen Inklusion.
	
	\textbf{Zu zeigen:} \( f^{-1}(V_1 \cap V_2) \subseteq f^{-1}(V_1) \cap f^{-1}(V_2) \) und \( f^{-1}(V_1) \cap f^{-1}(V_2) \subseteq f^{-1}(V_1 \cap V_2) \).
	
	\textbf{1. Inklusion:} Sei \( x \in f^{-1}(V_1 \cap V_2) \). Dann gilt:
	\begin{align*}
		f(x) & \in V_1 \cap V_2                                             \\
		     & \Rightarrow f(x) \in V_1 \text{ und } f(x) \in V_2           \\
		     & \Rightarrow x \in f^{-1}(V_1) \text{ und } x \in f^{-1}(V_2) \\
		     & \Rightarrow x \in f^{-1}(V_1) \cap f^{-1}(V_2)
	\end{align*}
	
	\textbf{2. Inklusion:} Sei \( x \in f^{-1}(V_1) \cap f^{-1}(V_2) \). Dann gilt:
	\begin{align*}
		x & \in f^{-1}(V_1) \text{ und } x \in f^{-1}(V_2)     \\
		  & \Rightarrow f(x) \in V_1 \text{ und } f(x) \in V_2 \\
		  & \Rightarrow f(x) \in V_1 \cap V_2                  \\
		  & \Rightarrow x \in f^{-1}(V_1 \cap V_2)
	\end{align*}
	
	Da \( x \) in beiden Fällen zu \( f^{-1}(V_1 \cap V_2) \) gehört, folgt die Gleichheit der Mengen:
	\[ f^{-1}(V_1 \cap V_2) = f^{-1}(V_1) \cap f^{-1}(V_2) \]
\end{proof}

\subsection*{(c) \( f(U_1 \cup U_2) = f(U_1) \cup f(U_2) \)}

\begin{proof}
	Wir zeigen die Gleichheit der Mengen durch Nachweis der gegenseitigen Inklusion.

	\textbf{Zu zeigen:} \( f(U_1 \cup U_2) \subseteq f(U_1) \cup f(U_2) \) und \( f(U_1) \cup f(U_2) \subseteq f(U_1 \cup U_2) \).

	\textbf{1. Inklusion:} Sei \( y \in f(U_1 \cup U_2) \). Dann existiert ein \( x \in U_1 \cup U_2 \), so dass \( f(x) = y \). Es folgt:
	\begin{align*}
		x & \in U_1 \cup U_2                                          \\
		  & \Rightarrow x \in U_1 \text{ oder } x \in U_2             \\
		  & \Rightarrow f(x) \in f(U_1) \text{ oder } f(x) \in f(U_2) \\
		  & \Rightarrow y \in f(U_1) \cup f(U_2)
	\end{align*}

	\textbf{2. Inklusion:} Sei \( y \in f(U_1) \cup f(U_2) \). Es folgt:
	\begin{align*}
		y & \in f(U_1) \text{ oder } y \in f(U_2)                                                       \\
		  & \Rightarrow \exists x_1 \in U_1 : f(x_1) = y \text{ oder } \exists x_2 \in U_2 : f(x_2) = y \\
		  & \Rightarrow \exists x \in U_1 \cup U_2 : f(x) = y                                           \\
		  & \Rightarrow y \in f(U_1 \cup U_2)
	\end{align*}

	Da \( y \) in beiden Fällen zu \( f(U_1 \cup U_2) \) gehört, folgt die Gleichheit der Mengen:
	\[ f(U_1 \cup U_2) = f(U_1) \cup f(U_2) \]
\end{proof}

\subsection*{(d) \( f(U_1 \cap U_2) \subseteq f(U_1) \cap f(U_2) \)}

\begin{proof}
	Wir zeigen, dass jedes Element der Bildmenge von \( U_1 \cap U_2 \) auch in der Schnittmenge der Bildmengen von \( U_1 \) und \( U_2 \) liegt.
	
	\textbf{Zu zeigen:} \( f(U_1 \cap U_2) \subseteq f(U_1) \cap f(U_2) \).
	
	Sei \( y \in f(U_1 \cap U_2) \). Dann existiert ein \( x \in U_1 \cap U_2 \) so, dass \( f(x) = y \). Da \( x \) sowohl in \( U_1 \) als auch in \( U_2 \) liegt, gilt:
	\begin{align*}
		x & \in U_1 \cap U_2                                         \\
		  & \Rightarrow x \in U_1 \text{ und } x \in U_2             \\
		  & \Rightarrow f(x) \in f(U_1) \text{ und } f(x) \in f(U_2) \\
		  & \Rightarrow y \in f(U_1) \text{ und } y \in f(U_2)       \\
		  & \Rightarrow y \in f(U_1) \cap f(U_2)
	\end{align*}
	
	Daher ist jedes Element von \( f(U_1 \cap U_2) \) auch ein Element von \( f(U_1) \cap f(U_2) \), und somit ist \( f(U_1 \cap U_2) \subseteq f(U_1) \cap f(U_2) \) bewiesen.
\end{proof}


\subsection*{(e) Ist \( f \) injektiv, so gilt \( f(U_1 \cap U_2) = f(U_1) \cap f(U_2) \)}

\begin{proof}
	Da \( f \) injektiv ist, hat jedes Element in \( N \) höchstens ein Urbild in \( M \). Wir zeigen die Gleichheit der Mengen durch Nachweis der gegenseitigen Inklusion.
	
	\textbf{Zu zeigen:} \( f(U_1 \cap U_2) \subseteq f(U_1) \cap f(U_2) \) und \( f(U_1) \cap f(U_2) \subseteq f(U_1 \cap U_2) \).
	
	Die erste Inklusion \( f(U_1 \cap U_2) \subseteq f(U_1) \cap f(U_2) \) haben wir bereits in Teil (d) bewiesen.
	
	Für die umgekehrte Inklusion, sei \( y \in f(U_1) \cap f(U_2) \). Dann existieren \( x_1 \in U_1 \) und \( x_2 \in U_2 \) so, dass \( f(x_1) = y \) und \( f(x_2) = y \). Da \( f \) injektiv ist, folgt \( x_1 = x_2 \). Also liegt \( x_1 \) (welches gleich \( x_2 \) ist) sowohl in \( U_1 \) als auch in \( U_2 \), d.h. \( x_1 \in U_1 \cap U_2 \). Daher ist \( y = f(x_1) \in f(U_1 \cap U_2) \).
	
	Somit ist \( f(U_1) \cap f(U_2) \subseteq f(U_1 \cap U_2) \) und zusammen mit der ersten Inklusion folgt die Gleichheit der Mengen:
	\[ f(U_1 \cap U_2) = f(U_1) \cap f(U_2) \]
\end{proof}

\subsection*{Gegenbeispiel zur Aussage (e)}

Um zu zeigen, dass auf die Injektivität von \( f \) in Teil (e) nicht verzichtet werden kann, geben wir ein Beispiel einer nicht-injektiven Abbildung \( f : \mathbb{Z} \rightarrow \mathbb{Z} \) und Mengen \( U_1, U_2 \subset \mathbb{Z} \) an, für die gilt: \( f(U_1 \cap U_2) \neq f(U_1) \cap f(U_2) \).

Betrachten wir die Funktion \( f : \mathbb{Z} \rightarrow \mathbb{Z}, f(x) = x^2 \). Diese Funktion ist offensichtlich nicht injektiv, da \( f(x) = f(-x) \) für alle \( x \in \mathbb{Z} \).

Wählen wir \( U_1 = \{1\} \) und \( U_2 = \{-1\} \), dann erhalten wir:

\begin{itemize}
	\item \( U_1 \cap U_2 = \emptyset \), also \( f(U_1 \cap U_2) = \emptyset \).
	\item \( f(U_1) = \{1^2\} = \{1\} \) und \( f(U_2) = \{-1^2\} = \{1\} \), also \( f(U_1) \cap f(U_2) = \{1\} \).
\end{itemize}

Es folgt, dass \( f(U_1 \cap U_2) = \emptyset \) nicht gleich \( f(U_1) \cap f(U_2) = \{1\} \) ist. Dieses Beispiel zeigt, dass ohne die Injektivität von \( f \) die Gleichheit in Aussage (e) nicht gewährleistet ist.

\section*{Aufgabe 7}

\subsection*{(a) Beweise zu Injektivität und Surjektivität}

\textbf{Satz 1.10 (Teilaussagen):} Für Abbildungen \( f : L \rightarrow M \) und \( g : M \rightarrow N \) gilt:

\begin{enumerate}
	\item Sind \( f \) und \( g \) surjektiv, so ist auch \( g \circ f \) surjektiv.
	\item Ist \( g \circ f \) injektiv, so ist auch \( f \) injektiv.
\end{enumerate}

\begin{proof}
	\textbf{Teil 1:} Wir nehmen an, dass \( f \) und \( g \) surjektiv sind. Für jedes Element \( n \) in \( N \) existiert aufgrund der Surjektivität von \( g \) ein Element \( m \) in \( M \) mit \( g(m) = n \). Da \( f \) ebenfalls surjektiv ist, existiert für dieses \( m \) ein Element \( l \) in \( L \) mit \( f(l) = m \). Folglich gilt für die Verkettung \( g \circ f \), dass \( (g \circ f)(l) = g(f(l)) = g(m) = n \). Da \( n \) beliebig gewählt war, ist \( g \circ f \) surjektiv.

	\textbf{Teil 2:} Wir nehmen nun an, dass \( g \circ f \) injektiv ist. Angenommen, es gibt \( l_1, l_2 \in L \) mit \( f(l_1) = f(l_2) \). Dann folgt \( (g \circ f)(l_1) = g(f(l_1)) = g(f(l_2)) = (g \circ f)(l_2) \). Da \( g \circ f \) injektiv ist, muss \( l_1 = l_2 \) gelten. Dies zeigt, dass \( f \) injektiv ist.
\end{proof}

\subsection*{(b) Gegenbeispiele zur Umkehrung v. Aussagen aus Teil (a)}

\textbf{Ziel:} Wir zeigen, dass die Umkehrungen der Aussagen aus Teil (a) nicht allgemeingültig sind, indem wir geeignete Gegenbeispiele konstruieren.

\begin{enumerate}
	\item \textbf{Gegenbeispiel für die Surjektivität:}
	      Wir definieren die Funktionen \( f : L \rightarrow M \) und \( g : M \rightarrow N \) wie folgt:
	      \begin{itemize}
		      \item Sei \( L = \{1\} \), \( M = \{a, b\} \), und \( N = \{\alpha\} \).
		      \item Die Funktion \( f \) ist gegeben durch \( f(1) = a \).
		      \item Die Funktion \( g \) ist gegeben durch \( g(a) = g(b) = \alpha \).
	      \end{itemize}
	      Obwohl \( f \) nicht surjektiv ist, da kein Element in \( L \) auf \( b \) abgebildet wird, ist die Verkettung \( g \circ f \) surjektiv, da für jedes Element in \( N \) ein Urbild in \( L \) existiert, nämlich \( 1 \).

	\item \textbf{Gegenbeispiel für die Injektivität:}
	      Wir betrachten die Mengen und Funktionen:
	      \begin{itemize}
		      \item Sei \( L = \{1, 2\} \), \( M = \{a, b\} \), und \( N = \{\alpha\} \).
		      \item Die Funktion \( f \) ist definiert durch \( f(1) = a \) und \( f(2) = b \), und ist somit injektiv.
		      \item Die Funktion \( g \) ist definiert durch \( g(a) = g(b) = \alpha \).
	      \end{itemize}
	      Hier ist \( f \) injektiv, aber die Verkettung \( g \circ f \) ist nicht injektiv, da sowohl \( f(1) \) als auch \( f(2) \) auf das gleiche Element \( \alpha \) in \( N \) abgebildet werden.
\end{enumerate}

Diese Gegenbeispiele zeigen, dass die Umkehrungen der Aussagen aus Teil (a) nicht zutreffen und somit die Originalaussagen nicht umkehrbar sind.

\section*{Aufgabe 8}

Gegeben ist der logische Ausdruck:
\[
	([ \neg (A \lor B)] \oplus [C \land ( \neg D)]) \rightarrow [A \land(C \lor D) \land ( \neg B)]
\]

\subsubsection*{(a) Belegung: \( (A,B,C,D) = (1,0,1,0) \)}

\begin{align*}
	\neg (A \lor B)                                 & \Rightarrow \neg (1 \lor 0) \Rightarrow \neg 1 \Rightarrow 0                                        \\
	C \land (\neg D)                                & \Rightarrow 1 \land (\neg 0) \Rightarrow 1 \land 1 \Rightarrow 1                                    \\
	A \land (C \lor D) \land (\neg B)               & \Rightarrow 1 \land (1 \lor 0) \land (\neg 0) \Rightarrow 1 \land 1 \land 1 \Rightarrow 1           \\
	[ \neg (A \lor B)] \oplus [C \land ( \neg D)]   & \Rightarrow 0 \oplus 1 \Rightarrow 1                                                                \\
	([ \neg (A \lor B)] \oplus [C \land ( \neg D)]) & \rightarrow [A \land(C \lor D) \land ( \neg B)] \Rightarrow 1 \rightarrow 1 \Rightarrow \text{WAHR}
\end{align*}

Daher ist der Wahrheitswert der Aussage für die \textbf{Belegung (a) WAHR}.

\subsubsection*{(b) Belegung: \( (A,B,C,D) = (0,1,1,0) \)}

\begin{align*}
	\neg (A \lor B)                                 & \Rightarrow \neg (0 \lor 1) \Rightarrow \neg 1 \Rightarrow 0                                          \\
	C \land (\neg D)                                & \Rightarrow 1 \land (\neg 0) \Rightarrow 1 \land 1 \Rightarrow 1                                      \\
	A \land (C \lor D) \land (\neg B)               & \Rightarrow 0 \land (1 \lor 0) \land (\neg 1) \Rightarrow 0 \land 1 \land 0 \Rightarrow 0             \\
	[ \neg (A \lor B)] \oplus [C \land ( \neg D)]   & \Rightarrow 0 \oplus 1 \Rightarrow 1                                                                  \\
	([ \neg (A \lor B)] \oplus [C \land ( \neg D)]) & \rightarrow [A \land(C \lor D) \land ( \neg B)] \Rightarrow 1 \rightarrow 0 \Rightarrow \text{FALSCH}
\end{align*}

Daher ist der Wahrheitswert der Aussage für die \textbf{Belegung (b) FALSCH}.

\end{document}
