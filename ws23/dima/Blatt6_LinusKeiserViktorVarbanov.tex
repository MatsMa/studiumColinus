\documentclass{article}
\usepackage{amsmath}
\usepackage{mathtools}
\usepackage{amssymb}
\usepackage{amsthm}
\usepackage{graphicx}
\usepackage[ngerman]{babel}

% "Beweis" anstatt "Proof" bei proof Umgebung
\renewcommand{\proofname}{Beweis}
% Ohne serielle Nummerierung
% \newtheorem*{theorem}{Satz}
% \newtheorem*{lemma}[theorem]{Lemma}

\title{Lösungen zu Übungsaufgaben 06 \\ \small Gruppe: Do 08-10 HS 3, Runa Pflume}
\author{Linus Keiser, Viktor Varbanov}
\date{\today}

\begin{document}

\maketitle


\section*{Aufgabe 1}

\subsection*{i) \( \text{ggT}(225, 162) \)}

\begin{align*}
	225 & = 162 \cdot 1 + 63, \\
	162 & = 63 \cdot 2 + 36,  \\
	63  & = 36 \cdot 1 + 27,  \\
	36  & = 27 \cdot 1 + 9,   \\
	27  & = 9 \cdot 3 + 0.
\end{align*}

Da der letzte Rest, der nicht null ist, 9 ist, gilt \( \text{ggT}(225, 162) = 9 \). Nun finden wir die Koeffizienten \( x \) und \( y \), sodass \( 9 = x \cdot 225 + y \cdot 162 \):

\begin{align*}
	9 & = 36 - 27 \cdot 1,                            \\
	9 & = 36 - (63 - 36 \cdot 1) \cdot 1,             \\
	9 & = 36 \cdot 2 - 63 \cdot 1,                    \\
	9 & = (162 - 63 \cdot 2) \cdot 2 - 63 \cdot 1,    \\
	9 & = 162 \cdot 2 - 63 \cdot 3,                   \\
	9 & = 162 \cdot 2 - (225 - 162 \cdot 1) \cdot 3,  \\
	9 & = -225 \cdot 3 + 162 \cdot 5,                 \\
	9 & = -225 \cdot 3 + (225 \cdot 1 + 162) \cdot 5, \\
	9 & = 225 \cdot (-3 + 5) + 162 \cdot 5,           \\
	9 & = 225 \cdot 2 + 162 \cdot 5.
\end{align*}

Also ist die gesuchte Linearkombination \( 9 = (-5) \cdot 225 + 7 \cdot 162 \). Die Koeffizienten \( x \) und \( y \) sind somit \( x = -5 \) und \( y = 7 \).

\subsection*{ii) \& iii)}

Die Rechenwege für die weiteren Zahlenpaare (144, 100) und (1909, 1660) führen wir analog durch und es ergeben sich folgende Linearkombinationen:

Für \( \text{ggT}(144, 100) \):
\[ 4 = (-9) \cdot 144 + 13 \cdot 100. \]

Für \( \text{ggT}(1909, 1660) \):
\[ 83 = 7 \cdot 1909 - 8 \cdot 1660. \]


\section*{Aufgabe 2}

Wir untersuchen die ganzzahligen Lösungen der folgenden Gleichungen:

\begin{enumerate}
	\item \(6x + 8y = 38\)
	\item \(65x - 26y = 91\)
	\item \(522x - 132y = 8\)
\end{enumerate}

\subsection*{Lösung für Gleichung (i)}
Zuerst berechnen wir den ggT von \(6\) und \(8\):
\begin{align*}
	\text{ggT}(6, 8) & = 2.
\end{align*}
Da \(2\) die Zahl \(38\) teilt, existiert eine Lösung.

Mit dem erweiterten Euklidischen Algorithmus erhalten wir die Bézout-Koeffizienten:
\begin{align*}
	-1 \cdot 6 + 1 \cdot 8 & = 2.
\end{align*}
Multipliziert mit \(\frac{38}{2} = 19\) erhalten wir die spezifische Lösung:
\begin{align*}
	x & = -19, \\
	y & = 19.
\end{align*}

\subsection*{Lösung für Gleichung (ii)}
Analog erhalten wir für \(65x - 26y = 91\):
\begin{align*}
	\text{ggT}(65, 26) & = 13.
\end{align*}
Auch hier teilt der ggT die Zahl \(91\), also gibt es eine Lösung.

Die Bézout-Koeffizienten sind:
\begin{align*}
	1 \cdot 65 - 2 \cdot 26 & = 13.
\end{align*}
Multipliziert mit \(\frac{91}{13} = 7\) ergibt sich:
\begin{align*}
	x & = 7,  \\
	y & = 14.
\end{align*}

\subsection*{Analyse von Gleichung (iii)}
Für die Gleichung \(522x - 132y = 8\) finden wir:
\begin{align*}
	\text{ggT}(522, 132) & = 6.
\end{align*}
Da \(6\) nicht \(8\) teilt, gibt es keine ganzzahligen Lösungen für diese Gleichung.

\section*{Aufgabe 3}

\textbf{Behauptung:} Für alle Zahlen \( a, b \in \mathbb{Z} \), die nicht beide gleich 0 sind, gilt
\[
	\text{ggT}\left( \frac{a}{\text{ggT}(a, b)}, \frac{b}{\text{ggT}(a, b)} \right) = 1.
\]

\begin{proof}
	Sei \( d = \text{ggT}(a, b) \). Dann existieren ganze Zahlen \( m \) und \( n \) mit \( a = md \) und \( b = nd \).

	Betrachten wir nun die Ausdrücke \( \frac{a}{\text{ggT}(a, b)} = \frac{md}{d} = m \) und \( \frac{b}{\text{ggT}(a, b)} = \frac{nd}{d} = n \).

	Es bleibt zu zeigen, dass \( \text{ggT}(m, n) = 1 \). Angenommen, dies ist nicht der Fall und es existiert ein Teiler \( t > 1 \), der sowohl \( m \) als auch \( n \) teilt. Dann teilt \( t \) ebenfalls \( md \) und \( nd \), d.h. \( a \) und \( b \), was im Widerspruch zur Definition von \( d \) als dem größten gemeinsamen Teiler von \( a \) und \( b \) steht. Daher muss die Annahme falsch sein, und es folgt \( \text{ggT}(m, n) = 1 \).

	Somit ist bewiesen, dass \( \text{ggT}\left( \frac{a}{\text{ggT}(a, b)}, \frac{b}{\text{ggT}(a, b)} \right) = 1 \) für alle \( a, b \in \mathbb{Z} \), die nicht beide gleich 0 sind.
\end{proof}

\section*{Aufgabe 4}

\textbf{Teil (i):} Beweis der Teilbarkeitsregel durch 11.

\textbf{Behauptung:} Eine Zahl \( a = a_n \cdot 10^n + a_{n-1} \cdot 10^{n-1} + \ldots + a_1 \cdot 10^1 + a_0 \) mit \( a_0, \ldots, a_n \in \{0, \ldots, 9\} \) ist genau dann durch 11 teilbar, wenn ihre alternierende Quersumme \( \sum_{i=0}^{n} (-1)^i a_i \) durch 11 teilbar ist.

\begin{proof}
	Zunächst beobachten wir, dass \( 10 \equiv -1 \) (mod 11), woraus folgt, dass \( 10^k \equiv (-1)^k \) (mod 11) für alle ganzen Zahlen \( k \).

	Daher kann die Zahl \( a \) umgeschrieben werden als:
	\[
		a \equiv a_n \cdot (-1)^n + a_{n-1} \cdot (-1)^{n-1} + \ldots + a_1 \cdot (-1)^1 + a_0 \quad (\text{mod} \; 11).
	\]
	Diese Darstellung entspricht der alternierenden Quersumme von \( a \).

	Somit ist \( a \) genau dann durch 11 teilbar, wenn \( a \equiv 0 \) (mod 11), was äquivalent dazu ist, dass die alternierende Quersumme durch 11 teilbar ist.
\end{proof}

\textbf{Teil (ii):} Anwendung der Regel auf 82199295211.

\textbf{Behauptung:} Die Zahl 82199295211 ist durch 11 teilbar.

\begin{proof}
	Wir berechnen die alternierende Quersumme von 82199295211:
	\[
		1 - 1 + 2 - 5 + 9 - 2 + 9 - 9 + 1 - 2 + 8 = 11.
	\]
	Da 11 durch 11 teilbar ist, bestätigt dies nach Teil (i), dass 82199295211 ebenfalls durch 11 teilbar ist.
\end{proof}

\section*{Zusatzaufgabe 5}

Gegeben: Ein Alphabet mit 26 Buchstaben, davon 5 Vokale und 21 Konsonanten.

\subsection*{(i) Wörter mit 3 verschiedenen Konsonanten und 2 verschiedenen Vokalen}
Wähle zuerst 3 Konsonanten und dann 2 Vokale aus, danach ordne alle 5 Buchstaben.
\[
	\binom{21}{3} \cdot \binom{5}{2} \cdot 5!
\]

\subsection*{(ii) Wörter, die den Buchstaben B enthalten}
B wird fixiert, wähle 2 weitere Konsonanten und 2 Vokale aus, ordne dann die verbleibenden 4 Buchstaben.
\[
	\binom{20}{2} \cdot \binom{5}{2} \cdot 4!
\]

\subsection*{(iii) Wörter, die B und C enthalten}
B und C werden fixiert, wähle einen weiteren Konsonanten und 2 Vokale aus, ordne dann die verbleibenden 3 Buchstaben.
\[
	\binom{19}{1} \cdot \binom{5}{2} \cdot 3!
\]

\subsection*{(iv) Wörter, die mit B beginnen und C enthalten}
B ist am Anfang fixiert, wähle 2 Konsonanten (inklusive C) und 2 Vokale aus, ordne die verbleibenden 4 Buchstaben.
\[
	\binom{20}{2} \cdot \binom{5}{2} \cdot 4!
\]

\subsection*{(v) Wörter, die mit B beginnen und mit C enden}
B am Anfang und C am Ende sind fixiert, wähle einen weiteren Konsonanten und 2 Vokale, ordne die verbleibenden 3 Buchstaben.
\[
	\binom{20}{1} \cdot \binom{5}{2} \cdot 3!
\]

\subsection*{(vi) Wörter, die A und B enthalten}
A und B werden fixiert, wähle 2 weitere Konsonanten und einen weiteren Vokal, ordne dann die verbleibenden 4 Buchstaben.
\[
	\binom{20}{2} \cdot \binom{4}{1} \cdot 4!
\]

\subsection*{(vii) Wörter, die mit A beginnen und B enthalten}
A ist am Anfang fixiert, wähle 2 Konsonanten (inklusive B) und einen weiteren Vokal, ordne die verbleibenden 4 Buchstaben.
\[
	\binom{20}{2} \cdot \binom{4}{1} \cdot 4!
\]

\subsection*{(viii) Wörter, die mit B beginnen und ein A enthalten}
B ist am Anfang fixiert, wähle 2 weitere Konsonanten und einen Vokal (inklusive A), ordne die verbleibenden 4 Buchstaben.
\[
	\binom{20}{2} \cdot \binom{4}{1} \cdot 4!
\]

\subsection*{(ix) Wörter, die mit A beginnen und mit B enden}
A am Anfang und B am Ende sind fixiert, wähle einen weiteren Konsonanten und einen weiteren Vokal, ordne die verbleibenden 3 Buchstaben.
\[
	\binom{20}{1} \cdot \binom{4}{1} \cdot 3!
\]

\subsection*{(x) Wörter, die A, B und C enthalten}
A, B und C werden fixiert, wähle einen weiteren Konsonanten und einen weiteren Vokal, ordne die verbleibenden 3 Buchstaben.
\[
	\binom{18}{1} \cdot \binom{4}{1} \cdot 3!
\]

\section*{Zusatzaufgabe 6}

Gegeben sind fünf identische, sechsseitige, faire Würfel.

\subsection*{(i) Genau ein Würfel zeigt eine Fünf}
Für die anderen vier Würfel gibt es jeweils 5 mögliche Ergebnisse (alles außer 5).
\[
	5^4
\]

\subsection*{(ii) Keiner der Würfel zeigt eine Eins}
Jeder Würfel hat 5 mögliche Ergebnisse (2, 3, 4, 5, 6).
\[
	5^5
\]

\subsection*{(iii) Alle sechs Augenzahlen werden angezeigt}
Dies ist unmöglich mit fünf Würfeln, daher ist die Anzahl der möglichen Ergebnisse 0.

\[
	0
\]

\subsection*{(iv) Die angezeigten Augenzahlen sind paarweise verschieden}
Wähle 5 verschiedene Zahlen aus den 6 möglichen aus. Die Reihenfolge ist irrelevant, da die Würfel identisch sind.
\[
	\binom{6}{5}
\]

\subsection*{(v) Die Würfel ergeben ein Full-House}
Wähle 2 verschiedene Zahlen aus den 6 möglichen aus. Eine Zahl wird dreimal, die andere zweimal angezeigt.
\[
	\binom{6}{2}
\]

\section*{Zusatzaufgabe 7}

Wir betrachten die zwei Personen, die nebeneinandersitzen wollen, als eine Einheit. Dadurch reduziert sich das Problem auf das Setzen von 4 Einheiten. Für diese 4 Einheiten gibt es \(3!\) mögliche Anordnungen. Innerhalb der Einheit, die die zwei Personen bildet, gibt es \(2!\) mögliche Anordnungen, da diese beiden Personen auf zwei Arten nebeneinandersitzen können.

Die Gesamtzahl der Anordnungen ergibt sich aus dem Produkt der Anordnungen der Einheiten und der Anordnungen innerhalb der Einheit:

\[
	3! \cdot 2! = 6 \cdot 2 = 12
\]

Folglich existieren insgesamt 12 unterschiedliche Möglichkeiten für die Anordnung der 5 Personen, wobei zwei bestimmte Personen nebeneinandersitzen müssen.

\end{document}
