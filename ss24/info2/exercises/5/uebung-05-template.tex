% Dokoumentenklassen legen das grobe Layout fest. Hier wird eine selbsterstellte Klasse verwendet, deswegen benötigt man zum erstellen des PDFs auch die Datei uebungsblatt.cls und uebungsblatt.sty
\documentclass{uebungsblatt}

\usepackage[linesnumbered,commentsnumbered]{algorithm2e}
\usepackage{tikz}
\usepackage{hyperref}
% Store \nl in \oldnl
\let\oldnl\nl
% Remove line number for one line
\newcommand{\nonl}{\renewcommand{\nl}{\let\nl\oldnl}}

%Hier wird die Kopfzeile erstellt
\header{\textbf{Grundlagen der Praktischen Informatik}\hfill\textbf{Sommersemester 2024}\\
Simon Keiser % Name des Gruppenmitglieds Eintragen
\hfill Georg-August-Universität Göttingen \\
Linus Keiser (16604467) % Name des Gruppenmitglieds Eintragen
\hfill Institut für Informatik\\ 
% Student:in3 (Matrikelnummer2)\\  % Name des Gruppenmitglieds Eintragen
% Student:in4 (Matrikelnummer2)\\  % Name des Gruppenmitglieds Eintragen

% Hiermit wird der horizontale Strick erzeugt, der die Kopfzeile abgrenzt.
\rule{\textwidth}{0.1mm}}

% Hier wird die Blattnummer festgelegt.
\blattnummer{5}

% Hier beginnt der eigentliche Textkörper. Alles zwischen \begin{document} und \end{document} ist der eigentliche Text
\begin{document}

% Mit \underline kann man Sachen unterstreichen

% \begin{aufgabe} erstellt ein Aufgabenumgebung mit [...] könnt ihr den Titel angeben und mit \score die Punktzahl festlegen. Ist aber für euch nicht so wichtig.
\begin{aufgabe}[Pumping Lemma \score{17}]
	Zeigen Sie, dass die Sprache
	$L = \{a^p \ | \ p \in \mathbb{N} \text{ ist eine Primzahl}\}$ nicht regulär ist.\\
	\score{17}

	\medskip
	\underline{Hinweise}

	\begin{itemize}
		\item
		      Es gibt unendlich viele Primzahlen, d.h. für jedes
		      $k \in \mathbb{N}$ gibt es eine Primzahl $p \geq k + 2$.
		\item
		      Der Widerspruch $xy^iz \not\in L$, d.h. $|xy^iz|$ ist keine Primzahl,
		      muss mit einem passend gewählten $i \in \mathbb{N}$ geführt werden.
	\end{itemize}

\end{aufgabe}
% Hier könnt ihr eure Lösung hinschreiben
\begin{loesung}
	Um zu zeigen, dass die Sprache $L = \{ a^p \mid p \in \mathbb{N} \text{ ist eine Primzahl} \} $ nicht regulär ist, verwenden wir das Pumping-Lemma für reguläre Sprachen. Das Pumping-Lemma besagt, dass für jede reguläre Sprache $L$ eine Konstante $p$ existiert, sodass jedes Wort $w \in L$ mit $|w| \geq p$ in drei Teile $w = xyz$ zerlegt werden kann, wobei folgende Bedingungen erfüllt sind:

	1. $|xy| \leq p$,
	2. $|y| \geq 1$,
	3. $xy^i z \in L$ für alle $i \in \mathbb{N}$.

	Nehmen wir an, $L$ sei regulär. Dann existiert eine Pumping-Länge $p$. Wählen wir ein Wort $w = a^q \in L$ mit $q$ eine Primzahl und $q \geq p$.

	Da $|w| = q \geq p$, können wir $w$ in drei Teile $w = xyz$ zerlegen, sodass $|xy| \leq p$ und $|y| \geq 1$. Da $|xy| \leq p$ und $w = a^q$, besteht $x$ und $y$ nur aus den Symbolen $a$. Also können wir schreiben:

	$$x = a^m, \quad y = a^n, \quad z = a^k$$

	mit $m + n \leq p$ und $n \geq 1$ und $m + n + k = q$.

	Nun betrachten wir das Wort $xy^2z$:

	$$xy^2z = a^m (a^n)^2 a^k = a^{m+n+n+k} = a^{q+n}$$

	Dabei ist $q+n$ die Länge des neuen Wortes. Da $n \geq 1$, ist $q+n > q$. Da $q$ eine Primzahl ist, ist $q+n$ keine Primzahl, weil eine Primzahl durch Addition einer positiven Zahl größer als 1 keine Primzahl mehr bleibt (es sei denn, die Zahl selbst ist wieder eine Primzahl, aber wir wissen, dass $n \geq 1$ und $q$ bereits so gewählt ist, dass $q+n$ keine Primzahl ist).

	Somit haben wir ein $i \in \mathbb{N}$ gefunden (nämlich $i = 2$), sodass $xy^i z \notin L$ ist. Dies steht im Widerspruch zur Annahme, dass $L$ regulär ist.

	Daraus folgt, dass $L$ nicht regulär sein kann. Q.E.D.

\end{loesung}

\newpage
%-----------------------------
\begin{aufgabe}[Grammatik \score{26}]
	\medskip
	Gegeben sei folgende Sprache über dem Alphabet $\Sigma = \{a,b\}$.
	\begin{align*}
		L = \{ w \in \Sigma^* \ | \ w\ \mbox{enthält weder das Teilwort $aa$ noch das Teilwort $bb$} \}
	\end{align*}

	\begin{enumerate}
		\item
		      \label{kurzeWoerter}
		      Geben Sie die sieben kürzesten Wörter aus $L$ an.\\
		      \score{4}
		\item
		      \label{abGrammatik}
		      Geben Sie eine reguläre Grammatik $G$ an, die $L$ erzeugt.
		      Benutzen Sie dabei höchstens 4 Nichterminale.\\
		      \score{8}
		\item
		      Zeigen Sie für die beiden längsten Wörter aus Aufgabenteil~\ref{kurzeWoerter}., jeweils
		      durch Angabe einer Ableitung, das diese Wörter zur Sprache $L(G)$ gehören, die
		      von der Grammatik aus Aufgabenteil \ref{abGrammatik}. erzeugt wird.\\
		      \score{4}
		\item
		      Ist $L$ eine reguläre Sprache? Mit Begründung.\\
		      \score{2}
		\item
		      Legen Sie eine natürliche Zahl $k \in \mathbb{N}$ fest und
		      geben Sie für jedes Wort $w \in L$ mit $|w| \ge k$ eine Zerlegung
		      $xyz \in \{a, b\}^*$ an, für die gilt
		      \begin{itemize}
			      \item
			            $w = xyz$,
			      \item
			            $|y| \ge 1$,
			      \item
			            $|xy| \le k$,
			      \item
			            für alle $i \in \mathbb{N} \cup \{0\}$ gilt, $xy^{i}z \in L$.
		      \end{itemize}
		      \score{8}
	\end{enumerate}
\end{aufgabe}
\begin{loesung}
	\begin{enumerate}

		\item
		      \begin{enumerate}
			      \item $\epsilon$ (das leere Wort)
			      \item $a$
			      \item $b$
			      \item $ab$
			      \item $ba$
			      \item $aba$
			      \item $bab$
		      \end{enumerate}

		      Diese Wörter erfüllen alle die Bedingung, dass sie weder das Teilwort \texttt{aa} noch \texttt{bb} enthalten.


		\item
		      Eine Grammatik $G$, die mit höchstens 4 Nichtterminalen $L$ erzeught, können wir wie folgt konstruieren:

		      \begin{itemize}
			      \item $N = \{S, A, B\}$
			      \item $T = \{a, b\}$
			      \item Produktionen:
			            \begin{align*}
				            S & \rightarrow aA \mid bB \mid \epsilon \\
				            A & \rightarrow bS \mid aA               \\
				            B & \rightarrow aS \mid bB
			            \end{align*}
			      \item Startsymbol: $S$
		      \end{itemize}

		      Mit diesen Produktionen ist sichergestellt, dass keine zwei aufeinanderfolgenden $a$- oder $b$-Symbole entstehen, da jede Produktion entweder das Symbol wechselt oder zum Startsymbol zurückkehrt, um neue Symbolfolgen zu erzeugen. Der $\epsilon$-Übergang erlaubt es, das Wort zu jedem Zeitpunkt zu beenden.


		      \label{gra}
		\item
		      Für das Wort $aba$:

		      \begin{enumerate}
			      \item $S$
			      \item $S \rightarrow aA$
			      \item $aA \rightarrow abS$
			      \item $abS \rightarrow abaA$
			      \item $abaA \rightarrow aba$ (durch Anwendung von $A \rightarrow \epsilon$)
		      \end{enumerate}

		      Für das Wort $bab$:

		      \begin{enumerate}
			      \item $S$
			      \item $S \rightarrow bB$
			      \item $bB \rightarrow baS$
			      \item $baS \rightarrow babB$
			      \item $babB \rightarrow bab$ (durch Anwendung von $B \rightarrow \epsilon$)
		      \end{enumerate}

		      Diese Ableitungen zeigen, dass die Wörter $aba$ und $bab$ in der von der Grammatik erzeugten Sprache $L(G)$ enthalten sind.


		\item
		      Wir wissen, dass $L = \{ w \in \{a, b\}^* \mid w \text{ enthält weder das Teilwort } aa \text{ noch das Teilwort } bb \} $ regulär ist, da eine reguläre Sprache eine Sprache ist, die von einem endlichen Automaten erkannt oder durch eine reguläre Grammatik erzeugt werden kann. In \ref{gra}. haben wir eine reguläre Grammatik $G$ konstruiert, die $L$ erzeugt. Zusätzlich können wir noch einen endlichen Automaten angeben, der $L$ akzeptiert.

		      Ein deterministischer endlicher Automat (DFA), der $L$ akzeptiert, kann wie folgt konstruiert werden:

		      \begin{itemize}
			      \item Zustände: $\{S, A, B, D\}$
			            \begin{itemize}
				            \item $S$: Startzustand und akzeptierender Zustand
				            \item $A$: Zustand, nachdem ein $a$ gelesen wurde
				            \item $B$: Zustand, nachdem ein $b$ gelesen wurde
				            \item $D$: Dead-Zustand (wird erreicht, wenn ein ungültiges Muster erkannt wird)
			            \end{itemize}
			      \item Alphabet: $\{a, b\}$
			      \item Übergangsfunktionen:
			            \begin{align*}
				            \delta(S, a) & = A \\
				            \delta(S, b) & = B \\
				            \delta(A, a) & = D \\
				            \delta(A, b) & = S \\
				            \delta(B, a) & = S \\
				            \delta(B, b) & = D \\
				            \delta(D, a) & = D \\
				            \delta(D, b) & = D
			            \end{align*}
			      \item Startzustand: $S$
			      \item Akzeptierende Zustände: $\{S, A, B\}$
		      \end{itemize}

		      Da ebenfalls ein deterministischer endlicher Automat existiert, der $L$ akzeptiert, ist bestimmt $L$ regulär. Q.E.D.

		\item
		      Wir wählen $k = 2$. Sei $w \in L$ mit $|w| \geq k$. Wir zerlegen $w$ wie folgt:

		      \begin{itemize}
			      \item Wähle $x = \epsilon$ (das leere Wort)
			      \item Wähle $y = w_1$ (das erste Zeichen von $w$)
			      \item Wähle $z = w_2w_3\ldots w_n$ (den Rest des Wortes ab dem zweiten Zeichen)
		      \end{itemize}

		      Formell: Sei $w = w_1w_2w_3\ldots w_n$ mit $w_1, w_2, \ldots, w_n \in \{a, b\}$ und $n \geq 2$. Dann ist die Zerlegung:
		      $$x = \epsilon, \quad y = w_1, \quad z = w_2w_3\ldots w_n$$
		      Überprüfung der Bedingungen:

		      \begin{enumerate}
			      \item $w = xyz$:

			            $$ w = \epsilon \cdot w_1 \cdot w_2w_3\ldots w_n = w_1w_2w_3\ldots w_n$$

			            Dies ist genau $w$.

			      \item $|y| \geq 1$:

			            $$ |y| = |w_1| = 1 \geq 1 $$

			      \item $|xy| \leq k$:

			            $$ |xy| = |\epsilon w_1| = |w_1| = 1 \leq 2 $$

			      \item Für alle $i \in \mathbb{N} \cup \{0\}$ gilt, $xy^i z \in L$:

			            Für jedes $i \in \mathbb{N} \cup \{0\}$ ist:

			            $$ xy^i z = \epsilon \cdot w_1^i \cdot w_2w_3\ldots w_n = w_1^i w_2w_3\ldots w_n $$

			            Da $w \in L$ ist, enthält $w$ weder $aa$ noch $bb$. Da $y = w_1$ nur ein einzelnes Zeichen $a$ oder $b$ ist und $w$ bereits die Bedingung erfüllt, dass weder $aa$ noch $bb$ vorkommen, bleibt diese Bedingung auch für $w_1^i w_2w_3\ldots w_n$ erfüllt.

			            \begin{itemize}
				            \item Wenn $w_1 = a$, dann ist $w_1^i = a^i$. Für alle $i \geq 0$ gibt es keine zwei aufeinanderfolgenden $a$ in $w_1^i w_2w_3\ldots w_n$, weil $w_2 \neq a$.
				            \item Wenn $w_1 = b$, dann ist $w_1^i = b^i$. Für alle $i \geq 0$ gibt es keine zwei aufeinanderfolgenden $b$ in $w_1^i w_2w_3\ldots w_n$, weil $w_2 \neq b$.
			            \end{itemize}
		      \end{enumerate}

		      Daher bleibt $xy^i z$ in der Sprache $L$ für alle $i \in \mathbb{N} \cup \{0\}$.

		      Mit $k = 2$ und der Zerlegung $x = \epsilon$, $y = w_1$, $z = w_2w_3\ldots w_n$ erfüllen wir alle Bedingungen des Pumping-Lemmas für die Sprache $L$.
	\end{enumerate}

\end{loesung}
\newpage
\begin{aufgabe}[Rechts-/Linkslineare Grammatik \score{16}]
	\medskip
	Gegeben sei folgende rechtslineare Grammatik $G = (N,T,P,S)$.
	\begin{itemize}
		\item
		      Nichtterminale
		      $N := \{$ START, BIN, NULL, OP $\}$
		\item
		      Terminale
		      $T := \{0, 1, \vee, \wedge\}$
		\item
		      Produktionen
		      \begin{tabbing}
			      $P$ := \{ \quad \= FACTOR \quad \= \kill
			      $P$ := \{ \> START  \>$\rightarrow$ 1 BIN $|$ 0 NULL $|$ 1 $|$ 0  \\
			      \> BIN    \>$\rightarrow$ 1 BIN $|$ 0 BIN $|$ $\vee$ OP $|$ $\wedge$ OP $|$ $\varepsilon$ \\
			      \> NULL   \>$\rightarrow$ $\vee$ OP $|$ $\wedge$ OP $|$ $\varepsilon$ \\
			      \> OP     \>$\rightarrow$ 1 BIN $|$ 0 NULL \quad \}
		      \end{tabbing}
		\item
		      Startsymbol $S :=$ START
	\end{itemize}

	\begin{enumerate}
		\item
		      \label{zweiWoerter}
		      Geben Sie zwei Worte der von $G$ erzeugten Sprache $L(G)$ an,
		      die jeweils mit $0$ und $1$ beginnen,
		      jeweils jedes Terminalsymbol mindestens einmal enthalten und
		      insgesamt keine Ziffernfolge mehr als einmal enthalten.\\
		      \score{2}
		\item
		      \label{linksGrammatik}
		      Geben Sie eine linkslineare Grammatik $G'$ an,
		      die dieselbe Sprache wie die rechtslineare Grammatik $G$
		      erzeugt, d.h. es gilt $L(G') = L(G)$.\\
		      \score{10}
		\item
		      Zeigen Sie für die beiden Wörter aus Aufgabenteil~\ref{zweiWoerter}., jeweils
		      durch Angabe einer Ableitung, das diese Wörter zur Sprache $L(G')$ gehören, die
		      von der Grammatik aus Aufgabenteil \ref{linksGrammatik}. erzeugt wird.\\
		      \score{4}

	\end{enumerate}
\end{aufgabe}
% Hier könnt ihr eure Lösung hinschreiben
\begin{loesung}

\end{loesung}
\end{document}
